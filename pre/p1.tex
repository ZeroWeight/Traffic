\documentclass[a4paper]{paper}
\usepackage{amsmath}
\usepackage{pdfpages}
\usepackage{graphicx}
\usepackage{wrapfig}
\usepackage{listings}
\usepackage{multicol}
\usepackage{color}
\usepackage{alltt}
\usepackage{marvosym}
% Style definition file generated by highlight 3.6, http://www.andre-simon.de/ 

% Highlighting theme: Print 

\newcommand{\hlstd}[1]{\textcolor[rgb]{0,0,0}{#1}}
\newcommand{\hlnum}[1]{\textcolor[rgb]{0,0,0}{#1}}
\newcommand{\hlesc}[1]{\textcolor[rgb]{0,0,0}{#1}}
\newcommand{\hlstr}[1]{\textcolor[rgb]{0,0,0}{#1}}
\newcommand{\hlpps}[1]{\textcolor[rgb]{0,0,0}{#1}}
\newcommand{\hlslc}[1]{\textcolor[rgb]{0.4,0.4,0.4}{\it{#1}}}
\newcommand{\hlcom}[1]{\textcolor[rgb]{0.4,0.4,0.4}{\it{#1}}}
\newcommand{\hlppc}[1]{\textcolor[rgb]{0,0,0}{\bf{#1}}}
\newcommand{\hlopt}[1]{\textcolor[rgb]{0,0,0}{\bf{#1}}}
\newcommand{\hllin}[1]{\textcolor[rgb]{0.53,0.53,0.53}{#1}}
\newcommand{\hlkwa}[1]{\textcolor[rgb]{0,0,0}{\bf{#1}}}
\newcommand{\hlkwb}[1]{\textcolor[rgb]{0,0,0}{\bf{#1}}}
\newcommand{\hlkwc}[1]{\textcolor[rgb]{0,0,0}{\bf{#1}}}
\newcommand{\hlkwd}[1]{\textcolor[rgb]{0,0,0}{\bf{#1}}}
\definecolor{bgcolor}{rgb}{1,1,1}


\newcommand{\tabincell}[2]{\begin{tabular}{@{}#1@{}}#2\end{tabular}}
\begin{document}
\section{A brief description of the algorithm}
\subsection{basic assumption and parameters}
Firstly, it is assumed that the physical quantity which is under control directly is the acceleration of the vehicle, while other physical quantities are set because of the theorem given by kinematics, such as 
$$v(t+\delta_t)=v(t)+a\Delta_t$$
$$x(t+\delta_t)=x(t)+v(t)\Delta_t+0.5a\Delta_t^2$$ 

However, the acceleration of the vehicle can not be controled precisely, a random noise distributed by normal dirtribution is added into the result given by all of the strategies, especially the ones may operated by human.

As for the speed limit,the platform considers the maximum absolute value of the acceleration of the vehicle is $a_{max}$, which is generally used in the acceleration process and tbe braking process. Also, the maximum speed limit is set to $v_{max}$, while the minimum speed limit is set to $v_{min}$.

No matter which strategy is chosen, if the acceleration given by the strategy is greater than the maximum acceleration $a_{max}$ or less than $-a_{max}$, or the calculated speed is greater than the maximum speed or less than the minimum speed, the program will automatically set these values to the boundary value. When the vehicle has to stop because of the red light, the minimum speed of the vehicle is set to 0, describing the parking behavior which may exist, however, the behavior of reversing is never allowed.

Also, we assume the expected speed of the vehicle, $v_{exp}$, is set to 80\% of the maximum speed limit.

In order to simpliy the simulation process, we assume that during the process of driving, there is no lane changing behavior. This hypothesis is also acceptable in the real life, because of the restrictions imposed by driving habits and traffic laws \& regulations, the vehicle should finish changing lane before enter the traffic junction. Therefore this assumption is reasonable. What's more, just from the perspective of traffic efficiency, lane-changing can only reduce the average efficiency of the traffic, therefore, as the primary inspection of the platform is efficiency, lane-changing situation is not taken into consideration.

Therefore, the car driving in the lane can be described as a queue, e.g. $Q_{in}$

\subsection{The Generation of the vehicles}
It is assumed that the arrival process of the vehicle is approximately a Poisson process, which is limited and adjusted by the following algorithm

Assuming that the intensity of the Poisson process is$\lambda$, the arrival time interval of the vehicle $S_n,S_{n-1}$can be expressed as 
$$P((S_n-S_{n-1}\le t) = 1 - e^{-\lambda t}$$
Because the simulation time scale is set to 0.1s, the average hourly traffic flow is $$\bar{F}=36000\lambda$$

For the traffic flow in one direction, after the average hourly traffic flow is set by the traffic flow adjustment slider described above, The vehicles in this direction will be generated according to the Poisson process with a specified intensity of $\lambda=\displaystyle{\frac{\bar{F}}{3600}}$, and will be pushed into the pending queue $Q_{pending}$in each direction.

After generating the vehicle, the next step is to consider which lane the vehicle belongs to, and it is assumed that the possibility of the vehicle appearing in each safe lane is same. Firstly, the condition to determine a safe lane is that the distance of last vehicle in the lane and line which the vehicles are generated ($S_{start}$) is greater than the safe distance $S_{safe}$, that is,
$$X_{-1}-S_{start}\le S_{safe}$$

Screening all lanes that meet the situation above in this direction, and place the vehicle in $Q_{pending}$ into one of these lanes randomly, while the position of the vehicle newly generated is set to $S_{start}$.

If there is no lanes suitable for placing the vehicle currently, the vehicles will remain in $Q_{pending}$ and wait for the next simulation session.

The generation of the vehicle is done in the second step of the simulation cycle, just following the process control model.
\subsection{The formation and evacuation of the queue in the intersection}
\label{section:feq}
To describe the phenomenon of queuing aat the intersection, a queue $Q_{block}$ is introduced to each lane towards the intersection. And according to the current signal phase and the length of the queue, all of the control strategies mentioned take these simple algorithm below.
\begin{enumerate}
\item Red Light

If the distance between the first vehicle in $Q_{in}$ and the last vehicle in $Q_{block}$ is less than the control range $S_{control}$, the first vehicle in $Q_{in}$ brake and is expected to stop just behind the last vehicle in $Q_{block}$. The expected distance between these two vehicles is defined as $S_{stop}$, describing the distance between the vehicles in the $Q_{block}$. 

If the $Q_{block}$ is empty and the distance between the first vehicle in $Q_{in}$ and the stopping line is less than $S_{control}$, the first vehicle in $Q_{in}$ brake and is expected to stop just on the stopping line$S_{end}$.

If the first vehicle in the $Q_{in}$is too far that neither of the two above are satisfied, the first vehicle in the $Q_{in}$ is supposed to drive freely.

Also, the vehicles in the $Q_{block}$ should stop and wait until the light truns green.
\item Green Light
distance between the first vehicle in $Q_{in}$ and the last vehicle in $Q_{block}$ (if it exists) should not be less than $S_{safe}$. And a braking is taken if the distance mentioned above is too short.
\end{enumerate}
Therefore, the brief idea of the algorithm above can be described as the fake code below\\ 

\noindent
\ttfamily
\hlstd{}\hllin{01\ }\hlkwa{if\ }\hlstd{}\hlopt{(}\hlstd{Q\textunderscore block}\hlopt{.}\hlstd{}\hlkwd{empty}\hlstd{}\hlopt{())\{}\\
\hllin{02\ }\hlstd{\ }\hlkwa{if\ }\hlstd{}\hlopt{(}\hlstd{Light\ }\hlopt{==\ }\hlstd{Green}\hlopt{)}\\
\hllin{03\ }\hlstd{}\hlstd{\ \ }\hlstd{Q\textunderscore in}\hlopt{.}\hlstd{}\hlkwd{first}\hlstd{}\hlopt{().}\hlstd{}\hlkwd{drive\textunderscore freely}\hlstd{}\hlopt{();}\\
\hllin{04\ }\hlstd{\ }\hlkwa{else}\\
\hllin{05\ }\hlstd{}\hlstd{\ \ }\hlstd{Q\textunderscore in}\hlopt{.}\hlstd{}\hlkwd{first}\hlstd{}\hlopt{().}\hlstd{}\hlkwd{brake\textunderscore to}\hlstd{}\hlopt{(}\hlstd{S\textunderscore end}\hlopt{);}\\
\hllin{06\ }\hlstd{}\hlopt{\}}\\
\hllin{07\ }\hlstd{}\hlkwa{else}\hlstd{}\hlopt{\{}\\
\hllin{08\ }\hlstd{\ }\hlkwa{if}\hlstd{}\hlopt{(}\hlstd{Q\textunderscore block}\hlopt{.}\hlstd{}\hlkwd{last}\hlstd{}\hlopt{().}\hlstd{pos}\hlopt{{-}}\hlstd{Q\textunderscore in}\hlopt{.}\hlstd{}\hlkwd{first}\hlstd{}\hlopt{().}\hlstd{pos}\hlopt{$<$\ }\hlstd{S\textunderscore control}\hlopt{)}\\
\hllin{09\ }\hlstd{}\hlstd{\ \ }\hlstd{Q\textunderscore in}\hlopt{.}\hlstd{}\hlkwd{first}\hlstd{}\hlopt{().}\hlstd{brake\textunderscore to\\
\hllin{10\ }}\hlstd{\ \ }\hlstd{}\hlopt{(}\hlstd{Q\textunderscore block}\hlopt{.}\hlstd{}\hlkwd{last}\hlstd{}\hlopt{().}\hlstd{pos}\hlopt{{-}}\hlstd{S\textunderscore stop}\hlopt{);\ }\\
\hllin{11\ }\hlstd{\ }\hlkwa{else}\\
\hllin{12\ }\hlstd{}\hlstd{\ \ }\hlstd{Q\textunderscore in}\hlopt{.}\hlstd{}\hlkwd{first}\hlstd{}\hlopt{().}\hlstd{}\hlkwd{drive\textunderscore freely}\hlstd{}\hlopt{();}\\
\hllin{13\ }\hlstd{}\hlopt{\}}\hlstd{}\\
\mbox{}
\normalfont
\normalsize


Where the brake\_to(desired\_pos) method means braking to a desired position.

Meanwhile, the growth and the dissipation of the queue $Q_{block}$ follow the method below.

\begin{enumerate}
\item Increase

If the first vehicle in $Q_{in}$ is closer than $S_{stop}$ from the last vehicle in $Q_{block}$ (it is set to $S_{end}$ if the $Q_{block}$ is empty), it is removed from the $Q_{in}$ and pushed into the $Q_{block}$

\item Dissipation

When the traffic light is green, all of the vehicles in the $Q_{block}$ is moving forward at the speed of $v_{dis}$, since the speed is relatively low, the acceleration process and any other phenomenon can be ignored. When the vehicle moves over the stopping lane and enters the intersection, it is removed from the $Q_{block}$
\end{enumerate}
And the method above can be described as the fake code below\\

\noindent
\ttfamily
\hlstd{}\hllin{01\ }\hlkwa{if\ }\hlstd{}\hlopt{(}\hlstd{light\ }\hlopt{==\ }\hlstd{green}\hlopt{)\{}\\
\hllin{02\ }\hlstd{\ Q\textunderscore block}\hlopt{.}\hlstd{}\hlkwd{all\textunderscore move\textunderscore forward}\hlstd{}\hlopt{(}\hlstd{v\textunderscore dis}\hlopt{)}\\
\hllin{03\ }\hlstd{}\hlopt{\}}\\
\hllin{04\ }\hlstd{}\hlkwa{else}\hlstd{}\hlopt{\{}\\
\hllin{05\ }\hlstd{\ }\hlkwa{if}\hlstd{}\hlopt{(!}\hlstd{Q\textunderscore block}\hlopt{.}\hlstd{}\hlkwd{empty}\hlstd{}\hlopt{())\{}\\
\hllin{06\ }\hlstd{}\hlstd{\ \ }\hlstd{}\hlkwa{if}\hlstd{}\hlopt{(}\hlstd{Q\textunderscore block}\hlopt{.}\hlstd{}\hlkwd{last}\hlstd{}\hlopt{().}\hlstd{pos}\hlopt{{-}}\hlstd{Q\textunderscore in}\hlopt{.}\hlstd{}\hlkwd{first}\hlstd{}\hlopt{().}\hlstd{pos\\
\hllin{07\ }}\hlstd{\ \ \ }\hlstd{}\hlopt{$<$\ }\hlstd{S\textunderscore stop}\hlopt{)\{}\\
\hllin{08\ }\hlstd{}\hlstd{\ \ \ }\hlstd{Vehicle\ v}\hlopt{;}\\
\hllin{09\ }\hlstd{}\hlstd{\ \ \ }\hlstd{v}\hlopt{=}\hlstd{Q\textunderscore in}\hlopt{.}\hlstd{}\hlkwd{getfirst}\hlstd{}\hlopt{();}\\
\hllin{10\ }\hlstd{}\hlstd{\ \ \ }\hlstd{Q\textunderscore block}\hlopt{.}\hlstd{}\hlkwd{push}\hlstd{}\hlopt{(}\hlstd{v}\hlopt{);}\\
\hllin{11\ }\hlstd{}\hlstd{\ \ }\hlstd{}\hlopt{\}}\\
\hllin{12\ }\hlstd{\ }\hlopt{\}}\\
\hllin{13\ }\hlstd{\ }\hlkwa{else}\hlstd{}\hlopt{\{}\\
\hllin{14\ }\hlstd{}\hlstd{\ \ }\hlstd{}\hlkwa{if}\hlstd{}\hlopt{(}\hlstd{S\textunderscore end}\hlopt{{-}}\hlstd{Q\textunderscore in}\hlopt{.}\hlstd{}\hlkwd{first}\hlstd{}\hlopt{().}\hlstd{pos}\hlopt{$<$\ }\hlstd{S\textunderscore stop}\hlopt{)\{}\\
\hllin{15\ }\hlstd{}\hlstd{\ \ \ }\hlstd{Vehicle\ v}\hlopt{;}\\
\hllin{16\ }\hlstd{}\hlstd{\ \ \ }\hlstd{v}\hlopt{=}\hlstd{Q\textunderscore in}\hlopt{.}\hlstd{}\hlkwd{getfirst}\hlstd{}\hlopt{();}\\
\hllin{17\ }\hlstd{}\hlstd{\ \ \ }\hlstd{Q\textunderscore block}\hlopt{.}\hlstd{}\hlkwd{push}\hlstd{}\hlopt{(}\hlstd{v}\hlopt{);}\\
\hllin{18\ }\hlstd{}\hlstd{\ \ }\hlstd{}\hlopt{\}}\\
\hllin{19\ }\hlstd{\ }\hlopt{\}}\\
\hllin{20\ }\hlstd{\ }\\
\hllin{21\ }\hlopt{\}}\hlstd{}\\
\mbox{}
\normalfont
\normalsize


All above describe the formation and evacuation of the queue in the intersection and its interaction with other models. The differences caused by the assumption can be ignored. Thanks to this model, it is easy for us to decouple the driving model and the queuing model, which makes it much easier to implement the other algorithm
\subsection{Basic models}
In the description of manual driving and automatic driving vehicle models, the car-following model and free driving model are frequently used models. Therefore, we will describe these models before discussing the control strategies.

\subsubsection{The implementation of the free-driving method}

When there is no speed guidance, a vehicle's driving behavior can be roughly classified into free-driving and car-following, which will be described in this section and the next section. $S_{control}$ is defined as the distance of interaction between vehicles. Thus, the two driving models are chosen according to the rules below.

\begin{enumerate}
\item When the distance to the front vehicle from the vehicle which we are concerned about is less than $S_{control}$, the driving choose the car-following strategy.

\item When the distance is greater than $S_{control}$, the driver choose the free-driving strategy, which means trying to reach the expected speed, $v_{exp}$, during this period, a smaller acceleration may be chosen if the current speed of vehicle is almost the same as $v_{exp}$, while a greater acceleration should be chosen if the current speed is too low, such as the vehicle is stopped.

Also, because some random noise may be introduced to the system, if the speed of the vehicle is higher than $v_{max}$, a minor deceleration should be taken.
\end{enumerate}
\subsubsection{The implementation of the car-following method}

As mentioned above, when the vehicles is close enough to each other, the car-following method is chosen. In the strategy we used, the commonly used driving psycho-physical model—Wiedemann model is adopted, that is,
$$a_n(t+\Delta_t)=\frac{[\Delta v_{n,n-1}(t)]^2}{2[\Delta x_{n,n-1}(t)-S_{exp}]}+a_{n-1}(t)$$

In the above equation, $S_{exp}$ represents the expected minimum safe-following distance. Since the speed of the vehicle in the intersection may have a big range, $S_{exp}$ should not be a constant. Therefore, according to the regulations on safe distance in Regulation on the Implementation of the Road Traffic Safety Law of the People's Republic of China, we use the linear correlation model of the minimum safe following distance and the speed of the front vehicle, therefore, the $S_{exp}$ can be described like this,
$$S_n(t)=\alpha v_{n-1}(t)+S_{safe}$$
While the $S_{safe}$is the same as the distance between the vehicles in \ref{section:feq}, and the constant $\alpha$ may be set by according to the experience.

Once the distance between the vehicles is less than $S_{exp}$, which means the distance between the vehicles is too short, an emergency brake is taken owing to the specific situation. In most cases, the acceleration of the vehicle is set to $0.5a_{max}$
\subsection{Manual driving model}
The plantform provides three models to verify the effectiveness of the platform, and the first model is manual driving model.

As the name of the model indicates, manual driving model is used to describe the behavior of a vehicle driving by human. This model is often used to provide a 'basic' traffic efficiency, while other control strategies are expected to behave better than this simple model.

Firstly, vehicles which are far (greater than $S_{inter}$) from the intersection are taken into consideration. Because they are so far from the intersection, they do not consider the influence of the traffic light. 

Therefore, these vehicles mat take the car-following method or the free-driving method. If a vehicle is the first one in $Q_{in}$, it has to take the the free-driving method.

As for the ones which are closer to the intersection, different method will be chosen to meet the different situation below.

\begin{enumerate}
\item Green traffic light with an empty waiting queue $Q_{block}$

In this situation, the behavior of the vehicles is similar to the behavior described above.

In addition to the behavior mentioned, the remain time of the green light is calculated, if the remaining time is less than $T_{safe}$, the possibility of entering the intersection during the green light may be taken into consider, if it is difficult for the vehicle to pass the stopping line before the light turns red, the vehicle may behave like the traffic light is red.

\item Green traffic light while the  waiting queue $Q_{block}$ is not empty

During this situation, the vehicles discussed may brake or accelerate to set its speed to $v_{dis}$, and try to enter the queue $Q_{block}$, once it enters the $Q_{block}$, it will move with the other vehicles in the queue.

Also, if the remaining time is less than $T_{safe}$, the program may consider slow down the vehicle and prepare to stop. 
\item Red traffic light

In this case, the vehicles discussed will behave just like what we have discussed in \ref{section:feq}
\end{enumerate}

What's more, we focus on portraying the behavior of the first vehicle in the driving queue $Q_{in}$, since the vehicles behind it can be controled easily by car-following method.

Taking all of the situations into consideration. the manual driving model can be abstract as the fake code below.\\
 
\noindent
\ttfamily
\hlstd{}\hllin{01\ }\hlslc{//the\ code\ below\ just\ consider\ the\ vehicle\ v}\\
\hllin{02\ }\hlstd{}\hlkwa{if}\hlstd{}\hlopt{(}\hlstd{S\textunderscore end}\hlopt{{-}}\hlstd{v}\hlopt{.}\hlstd{pos}\hlopt{$>$}\hlstd{S\textunderscore inter}\hlopt{)}\hlstd{}\hlslc{//far\ from\ intersection}\\
\hllin{03\ }\hlstd{}\hlopt{\{}\\
\hllin{04\ }\hlstd{\ }\hlslc{//code\ block\ 1}\\
\hllin{05\ }\hlstd{\ }\hlkwa{if}\hlstd{}\hlopt{(}\hlstd{car}\hlopt{{-}}\hlstd{}\hlkwd{following\textunderscore requirement\textunderscore met}\hlstd{}\hlopt{())}\\
\hllin{06\ }\hlstd{}\hlstd{\ \ }\hlstd{}\hlslc{//the\ car{-}following\ requirement\ is}\hlstd{\ \ }\hlslc{met}\\
\hllin{07\ }\hlstd{}\hlstd{\ \ }\hlstd{}\hlkwd{car\textunderscore following}\hlstd{}\hlopt{();}\\
\hllin{08\ }\hlstd{\ }\hlkwa{else}\\
\hllin{09\ }\hlstd{}\hlstd{\ \ }\hlstd{}\hlslc{//the\ head\ car\ of\ the\ queue\ or}\\
\hllin{10\ }\hlstd{}\hlstd{\ \ }\hlstd{}\hlslc{//the\ car{-}following\ requirement\ isn't\ met}\\
\hllin{11\ }\hlstd{}\hlstd{\ \ }\hlstd{}\hlkwd{free\textunderscore driving}\hlstd{}\hlopt{();}\\
\hllin{12\ }\hlstd{}\hlopt{\}}\\
\hllin{13\ }\hlstd{}\hlkwa{else\ }\hlstd{}\hlslc{//close\ to\ the\ intersection}\\
\hllin{14\ }\hlstd{}\hlopt{\{\ }\\
\hllin{15\ }\hlstd{\ }\hlkwa{if}\hlstd{}\hlopt{(}\hlstd{light}\hlopt{==}\hlstd{red}\hlopt{)}\\
\hllin{16\ }\hlstd{}\hlstd{\ \ }\hlstd{method1}\hlopt{.}\hlstd{}\hlkwd{3}\hlstd{}\hlopt{();}\hlstd{}\hlslc{//the\ method\ mentioned\ in\ 1.3}\\
\hllin{17\ }\hlstd{\ }\hlkwa{else}\\
\hllin{18\ }\hlstd{}\hlstd{\ \ }\hlstd{}\hlkwa{if}\hlstd{}\hlopt{(}\hlstd{Q\textunderscore block}\hlopt{.}\hlstd{}\hlkwd{empty}\hlstd{}\hlopt{())\{}\\
\hllin{19\ }\hlstd{}\hlstd{\ \ \ }\hlstd{}\hlkwa{if}\hlstd{}\hlopt{(}\hlstd{time\textunderscore remain}\hlopt{$>$}\hlstd{T\textunderscore safe}\hlopt{)}\\
\hllin{20\ }\hlstd{}\hlstd{\ \ \ }\hlstd{}\hlslc{//same\ as\ the\ code\ block\ 1}\\
\hllin{21\ }\hlstd{}\hlstd{\ \ \ }\hlstd{}\hlkwa{else\ }\\
\hllin{22\ }\hlstd{}\hlstd{\ \ \ \ }\hlstd{}\hlkwa{if}\hlstd{}\hlopt{(!}\hlstd{}\hlkwd{pass\textunderscore check}\hlstd{}\hlopt{())}\\
\hllin{23\ }\hlstd{}\hlstd{\ \ \ \ }\hlstd{}\hlslc{//same\ as\ the\ red\ light\ case}\\
\hllin{24\ }\hlstd{}\hlstd{\ \ \ }\hlstd{}\hlopt{\}}\\
\hllin{25\ }\hlstd{}\hlstd{\ \ }\hlstd{}\hlkwa{else}\\
\hllin{26\ }\hlstd{}\hlstd{\ \ \ }\hlstd{}\hlkwa{if}\hlstd{}\hlopt{(}\hlstd{time\textunderscore remain}\hlopt{$>$}\hlstd{T\textunderscore safe}\hlopt{)}\\
\hllin{27\ }\hlstd{}\hlstd{\ \ \ }\hlstd{}\hlkwd{acc\textunderscore to\textunderscore speed}\hlstd{}\hlopt{(}\hlstd{v\textunderscore dis}\hlopt{);}\\
\hllin{28\ }\hlstd{}\hlstd{\ \ \ }\hlstd{}\hlkwa{else\ }\\
\hllin{29\ }\hlstd{}\hlstd{\ \ \ \ }\hlstd{}\hlkwa{if}\hlstd{}\hlopt{(!}\hlstd{}\hlkwd{pass\textunderscore check}\hlstd{}\hlopt{())}\\
\hllin{30\ }\hlstd{}\hlstd{\ \ \ \ }\hlstd{}\hlslc{//same\ as\ the\ red\ light\ case}\\
\hllin{31\ }\hlstd{}\hlstd{\ \ \ }\hlstd{}\hlopt{\}}\\
\hllin{32\ }\hlstd{}\hlstd{\ \ }\hlstd{}\\
\hllin{33\ }\hlopt{\}}\hlstd{}\\
\mbox{}
\normalfont
\normalsize


The plantform will traverse the queue $Q_{in}$, and during each iteration, the acceleration of the vehicle in processing will be set according to the model given above.

\subsection{The single-vehicle cooperative speed guidance model}

\subsection{The multi-vehicle cooperative speed guidance model}
\end{document}