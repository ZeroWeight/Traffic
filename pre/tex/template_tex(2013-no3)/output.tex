\protect\hypertarget{ux5fToc470790476}{}{}\textbf{A Simulation System
and Speed Guidance Algorithms for Intersection Traffic Control Using
Connected Vehicle Technology}

\textbf{Abstract:} In the connected vehicle environment, it is possible
to obtain real-time vehicle-state data through vehicle-to-infrastructure
communication, and significantly increase the prediction accuracy of
urban traffic conditions. This study uses the C++/Qt programming
language and framework to build a simulation platform. A two-way
six-lane intersection was set up on the simulation platform. In
addition, two speed guidance algorithms basing on optimizing the travel
time of a single vehicle or multi vehicles respectively are proposed.
The goal of optimization is to minimize the travel time, with the common
indicators such as average delay of vehicles, average number of stops,
and average stop time chosen as indexes of traffic efficiency. The
simulation results show that when the traffic flow is not saturated,
compared with the case of no speed guidance, single-vehicle speed
guidance can improve the traffic efficiency by 20\%, whereas
multi-vehicle speed guidance can improve the traffic efficiency by 50\%.
When the traffic flow is saturated, the speed guidance algorithms have
performances that are more outstanding. With the increase of penetration
rate, the effect of speed guidance gradually enhances, with the most
obvious gains obtained when the penetration rate increases from 10\% to
40\%. Thus, this study has shown that speed guidance in the connected
vehicle environment can significantly improve the traffic efficiency of
intersections, and compared with the single-vehicle speed guidance
strategy, the multi-vehicle speed guidance strategy is more effective.

\textbf{Keywords:} connected vehicle; intersection traffic control;
simulation system; speed guidance

\textsc{INTRODUCTION}

The concept of cooperative driving was first introduced in the early
1990s. A cooperative driving vehicle can control its velocity and
trajectory to optimize objectives of traffic efficiency, e.g. average
delay and average stop time. In 2006, Li et al. proposed the concept of
``safety driving patterns'' to obtain the allowable movement schedules
of all vehicles which entered the intersection via traffic lights and
Vehicle-to-Infrastructure (V2I) communication \textsuperscript{{[}1{]}
{[}2{]}}. The connected vehicle technology has become one of the
cutting-edge technologies in the field of intelligent transportation
today. It is an effective way to improve traffic efficiency, increase
road safety, and reduce traffic pollution \textsuperscript{{[}3{]}}. The
connected-vehicle system has obvious positive effects on improving the
efficiency of traffic networks. When vehicles are connected to each
other, the drivers' reaction time can be shortened, and the headway
between vehicles can also be shortened, thus leading to higher road
occupancy and higher road capacity \textsuperscript{{[}4{]}}. Under the
connected-vehicle environment, the full-time acquisition of vehicle
state data can effectively improve the prediction accuracy of urban
traffic conditions \textsuperscript{{[}5{]}}. Using real-time V2I
communication instead of traditional detectors can improve the timely
response ability of the road traffic signal controller
\textsuperscript{{[}6{]}}. In addition, monitoring the states of
platoons can make the control and coordination of traffic signals more
accurate \textsuperscript{{[}6{]}}.

Nowadays, researchers have conducted several studies on speed guidance
under the connected-vehicle environment. Nekoui explored the issue of
road traffic safety by mathematical models and field experiments. They
revealed that introducing speed guidance under the connected-vehicle
environment could effectively relieve the problems of vehicle emergency
avoidance and collision avoidance under different conditions
\textsuperscript{{[}7{]}}. Malakorn and Park examined a cooperative
transportation system, which allowed vehicles to accept trajectory
instructions from an intelligent traffic signal using the two-phase
signal-timing plan, and found that it was highly beneficial in terms of
both mobility and fuel consumption \textsuperscript{{[}8{]}}. Abu-Lebdeh
analyzed the feasibility of dynamic speed control in his paper, and
discussed its potential benefits in the field of traffic control
\textsuperscript{{[}9{]}}.

Yang et al. studied main rural roads \textsuperscript{{[}10{]}}. They
proposed a speed guidance strategy considering factors such as the
location of vehicles, the status of signal controls, the acceleration
and deceleration time of vehicles, drivers' acceptance, and etc. They
further used the VISSIM software, a microscopic simulator, to simulate
the connected-vehicle environment in order to validate their speed
guidance strategy. Chen et al. used roadside variable message signs as
display terminals for speed guidance \textsuperscript{{[}11{]}}.
Considering the locations of roadside speed guidance equipment and
signal timing, they put forward a strategy which combines dynamic speed
guidance and dynamic signal control to optimize the arterial coordinated
signal control system. He used the real-time traffic data from V2V
(Vehicle-to-Vehicle) and V2I communication to establish a platoon
recognition algorithm basing on headways \textsuperscript{{[}12{]}}. In
addition, an optimized signal-control model named PAMSCOD (Platoon-based
arterial multi-modal signal control with online data) was proposed using
mixed integer linear programming for the coordination between several
arterial intersections \textsuperscript{{[}12{]}}. Lee and Park proposed
a cooperative vehicle control algorithm which minimized the total length
of overlapped trajectories to avoid potential collisions in the
intersection \textsuperscript{{[}13{]}}. Jackline and Andreas proposed a
close-form solution in a centralized fashion for cooperative-driving
vehicles to merge at expressway on-ramps \textsuperscript{{[}14{]}}.

Given the limited deployment of Cooperative Vehicle Infrastructure
Systems (CVIS), simulation study is a very important research method.
Currently, mature commercial traffic simulation software such as VISSIM,
Paramics, and TransModeler are widely used \textsuperscript{{[}15{]}}.
However, they cannot realize V2V or V2I communication, neither can they
realize real-time intervention of the running status of vehicles.

The smart CVIS of China is named as i-VICS (Intelligent
Vehicle-Infrastructure Cooperation Systems), which is the research
result of related studies in the recent ten years
\textsuperscript{{[}16-18{]}}. Basing on the technical characteristics
of i-VICS, this study designed an intersection simulation system in the
CVIS environment which realizes V2V and V2I communication and real-time
intervention of the running status of vehicles. In addition, two speed
guidance algorithms were proposed and tested in the self-developed
simulation system.

The remaining of this paper is organized as follows. Section 2 describes
the methodologies including the vehicle dynamics description and
speed-guidance algorithms. Section 3 introduces the simulation system we
designed in this research. A simulation-based case study and the
corresponding results are presented in Section 4. Finally, concluding
remarks are presented in Section 5.

METHODOLOGY

In this section, we use mathematical language to model the intersection
traffic control problem in the connected vehicle environment. Firstly,
the vehicle dynamics model is provided. Then, the way vehicles move
without speed guidance is described. Next, the two speed-guidance
algorithms we designed, named the single-vehicle speed guidance
algorithm and the multi-vehicle cooperative speed guidance algorithm,
are introduced. Finally yet importantly, the constraints of the system
are listed.

\begin{enumerate}
\def\labelenumi{\arabic{enumi}.}
\item
  \textbf{Vehicle Dynamics Description and Assumption}
\end{enumerate}

\begin{enumerate}
\def\labelenumi{\arabic{enumi})}
\item
  \begin{quote}
  Definition of variables
  \end{quote}
\end{enumerate}

For the convenience of expression, the main variables to be used are
defined in Table 1.

Table 1 The definition of main variables

\begin{longtable}[c]{@{}ll@{}}
\toprule
Variable & Definition\tabularnewline
\midrule
\endhead
& System state, including the traffic condition and the controlled
variables\tabularnewline
& Traffic condition, including the states of all the vehicles in the
current system\tabularnewline
& Functional parameters\tabularnewline
& Controlled variables, including the guided speed and the information
of the traffic light (Boolean variable, with 0 representing red light,
and 1 representing green light)\tabularnewline
& Vector of the vehicle state, including three dimensions of current
speed, location, and waiting time\tabularnewline
& Acceleration (assumed constant, positive when speeding up, and
negative when slowing down)\tabularnewline
& Distance from the current position to the stop-line\tabularnewline
& Travel time (the time interval between the current moment and the
moment leaving the stop-line)\tabularnewline
& Guided speed\tabularnewline
& Current speed\tabularnewline
& Total waiting time in the waiting area\tabularnewline
& Discount factor\tabularnewline
& Real value of the optimization function\tabularnewline
& One-step cost function\tabularnewline
\bottomrule
\end{longtable}

From the above definition, the system state can be expressed as , with
and defined as:

In the above equations, \emph{N} is the current total number of vehicles
in the system.

The vehicle state (vector) can be expressed as:

Whereas the controlled variable can be denoted as:

with

The simple dynamic equation of a vehicle passing the stop-line can be
expressed as:

For a constant acceleration , the above equation can be written as:

Thus, we obtain

\begin{quote}
For a vehicle which is in the stop state, the travel time can be
calculated through the following equation:
\end{quote}

With .

\begin{enumerate}
\def\labelenumi{\arabic{enumi})}
\item
  \begin{quote}
  Basic assumptions
  \end{quote}
\end{enumerate}

In order to simplify the research process, we made the following basic
assumptions in this research:

\begin{enumerate}
\def\labelenumi{\alph{enumi})}
\item
  The studied region is a single intersection, i.e. the influence of
  other intersections does not need to be considered.
\item
  The length of guiding region is 100m away from the stopping line in
  every direction.
\item
  \begin{quote}
  The signal at the intersection is controlled using fixed-cycle
  strategies.
  \end{quote}
\item
  \begin{quote}
  The vehicles have changed lanes before entering the controlled region,
  i.e. the vehicles will not change lanes in the controlled region.
  \end{quote}
\item
  The vehicles with on-board equipment will follow the speed-guidance
  strategies.
\item
  When the vehicles pass the stop-line and enter the intersection, they
  will return to the state of autonomous driving.
\end{enumerate}

\begin{enumerate}
\def\labelenumi{\arabic{enumi}.}
\item
  \textbf{Driving Behavior without speed guidance}
\end{enumerate}

When there is no speed guidance, a vehicle's straight driving behavior
can be classified into free driving and car following. We define 150
meters as the distance of interaction between vehicles, the same as that
of the VISSIM simulation system. Thus, the two driving behaviors are as
follows:

\begin{enumerate}
\def\labelenumi{\alph{enumi})}
\item
  When the distance to the front vehicle is equal to or greater than 150
  meters, the driver chooses the free driving strategy, and tries to
  reach the maximum speed (defined as 90\% of the speed limit of the
  road) as soon as possible.
\item
  When the distance to the front vehicle is less than 150 meters, the
  driver chooses the car following strategy. The commonly used driving
  psycho-physical model---Wiedemann model \textsuperscript{{[}19{]}} is
  adopted in this study.
\end{enumerate}

\begin{quote}
In the above equation, represents the expected minimum safe-following
distance. According to the regulations on safe distance in
\emph{Regulation on the Implementation of the Road Traffic Safety Law of
the People's Republic of China} \textsuperscript{{[}20{]}}, the linear
correlation model of the minimum safe following distance and the speed
of the front vehicle is used:
\end{quote}

In the above equation, is a linear coefficient, is the minimum headway
when stopped. For instance, when the speed of the front vehicle is 50
km/h, the minimum safe-following distance of the vehicle behind is 40 m.

\begin{enumerate}
\def\labelenumi{\arabic{enumi}.}
\item
  \textbf{Speed-guidance algorithms}
\end{enumerate}

\begin{enumerate}
\def\labelenumi{\arabic{enumi})}
\item
  The single-vehicle speed guidance algorithm
\end{enumerate}

Currently, in the driving process, most vehicles cannot acquire the
signal control information of the front intersection in real-time,
neither can the drivers judge in advance whether they can pass the
stop-line in the current signal cycle or not. Therefore, the drivers can
only rely on experience and driving habits to decide the speed of the
vehicle. With the emergence of connected-vehicle technology, vehicles
can acquire the real-time information of traffic conditions and signal
information, therefore it is possible to adjust vehicle speed through
speed guidance and to increase the traffic efficiency. If we chose a
vehicle as the object of speed guidance and set its minimum travel time
as the optimization objective, the speed-guidance process can be
described as:

\begin{enumerate}
\def\labelenumi{\alph{enumi})}
\item
  When the vehicle enters the guiding region, the signal is green, and
  there is no vehicle in the queue.
\end{enumerate}

At this time, our goal is to make the vehicle pass the stop-line as soon
as possible before the green light ends. Thus, the optimize function is:

and

is the moment that the green light was turned on, and is the duration of
the green light.

If there is no feasible solution (i.e.), it means that this vehicle
cannot pass the stop-line through speed guidance and have to enter the
queue, which can be classified into the next situation.

\begin{enumerate}
\def\labelenumi{\alph{enumi})}
\item
  When the vehicle enters the guided region, the signal is red; or the
  signal is green, but the queue has not dissipated.
\end{enumerate}

At this time, our goal is to make the vehicle arrive at the intersection
when the queue has just dissipated. Thus, the optimize function is:

and

\begin{quote}
is the loss of the first vehicle's boot time, is the current number of
vehicles in the queue, and is the saturation flow rate of this lane.

By solving the above objective function, we can obtain the speed control
strategy to be taken right now, that is, the acceleration that should be
accepted by the current vehicle.
\end{quote}

\begin{enumerate}
\def\labelenumi{\arabic{enumi})}
\item
  The multi-vehicle cooperative speed guidance algorithm
\end{enumerate}

According to the above description, the time that each vehicle needs to
pass the stop-line can be expressed by mathematical functions. Basing on
this, we can build the optimization function with the objective of
minimizing the total time needed by all vehicles to pass the stop-line.

First of all, the one-step cost function is defined as follows:

Its physical meaning is the difference of total time that all vehicles
needed to pass the stop-line between the two calculation moments.

Thereby, we can define the cost function as:

In the above equation, is the discount factor, which can help to obtain
the best effect as much as possible at the first step of optimization.

The optimization function of this optimization problem is:

\begin{quote}
Similarly, we can obtain the control strategy to be taken right now,
that is, the acceleration to be accepted by the current guided vehicle.
\end{quote}

\begin{enumerate}
\def\labelenumi{\arabic{enumi}.}
\item
  \textbf{Constraints}
\end{enumerate}

The constraints of the system are listed as follows:

\begin{enumerate}
\def\labelenumi{\alph{enumi})}
\item
  The time constraint of the `head vehicle' (the first vehicle in the
  queue) to pass the stop-line. After the green signal lights up, the
  time of the first vehicle behind the stop-line to pass the stop-line
  should be equal to or larger than the time that the green signal
  lights up:
\item
  The time constraint between the stop state and the booting process.
  After the green signal lights up, if the first vehicle behind the
  stop-line is in the stop state, the time it passes the stop-line
  should be equal to or larger than the sum of the time that the green
  light is turned on () and the loss of the first car's boot time ():
\item
  The time constraint for two consecutive vehicles to pass the
  stop-line. The time for the latter vehicle to pass the stop-line
  should be equal to or larger than the sum of the time that the vehicle
  before it passes the stop-line and the minimum headway ():
\item
  The constraint of optimized speed. The optimized speed calculated by
  the model should lie between the low and high thresholds of speed. If
  the optimized speed is not in this range, it means the vehicle has to
  stop and wait:
\item
  The constraint of the number of vehicles passing the stop-line. The
  number of vehicles passing the stop-line in one signal cycle should be
  smaller than or equal to the volume under saturated state:
\end{enumerate}

In the above equation, is the number of lanes in the studied direction,
and is the effective green time.

SIMULATION SYSTEM DESIGN

\begin{enumerate}
\def\labelenumi{\arabic{enumi}.}
\item
  \textbf{The Framework and Design of the Simulation System}
\end{enumerate}

We used the C++/Qt programming language and framework to build the
simulation system. As shown in Figure 1, the simulation system mainly
consists of three modules, i.e. the signal control module, the user
strategy module, and the core simulation module.

Figure 1 The framework of the traffic control simulation system under
the connected-vehicle environment

In the signal control module, an interface which can adjust the signal
cycle and the states of the 12 traffic lights in four directions (as
shown in Figure 2) is provided. The signal cycle and each light's state
change process can be preset before the simulation runs, thus providing
a traffic signal control plan for the whole simulation process.

In the user strategy module, users can have access to system information
such as the location, velocity, acceleration, etc. of a vehicle. As long
as the control strategy of vehicle acceleration is transmitted to the
simulation system, the control of vehicles in the simulation system can
be realized.

In the simulation module, a graphical interface of the operation program
(as shown in Figure 3) is provided. The current phase of each signal at
the intersection, and both the number and real-time position of each
vehicle are clearly displayed on the interface. The main parameters of
the simulation control module can be adjusted on the interface,
including the proportion of vehicles that are equipped with on-board
equipment, the traffic volume at each direction, the simulation speed,
and etc. The preliminary statistics data provided by the data analysis
module are also presented on the interface, e.g. the operation time of
the program and the number of vehicles in each lane that have passed the
stop-line. Whereas the processed data such as vehicle delay, number of
stops, etc. are directly exported into Excel files.

It should be specially explained that the simulation speed is the
running speed of the simulation system, i.e. the ratio between the time
in the simulation system and the time in real-life. There are four
simulation speeds in our simulation system: fast, medium, slow, and very
slow. When the simulation speed is `fast', the simulation system's run
speed is 100 times of that in the real world, which means that when the
system runs for one second, a vehicle in the simulation system has
already run for 100 seconds, thus greatly accelerating the speed of the
simulation experiment. When the simulation speed is `medium', the
simulation system's run speed is 10 times of that in the real world,
which is suitable for a rough observation of the system operation
status. `Slow' means that the simulation system has the same running
speed as that in the real world, which can most truly reflect the
vehicles' running states. `Very slow' means that the running speed in
the simulation system is 1/10 of that in the real world, which is
helpful for detailed observation of a certain vehicle's running status.
Therefore, the different simulation speeds can meet the needs of the
different studies. In addition, the simulation speeds can be
interchanged freely in continuous operation, which is very convenient.

The simulation system has good expandability. It can increase modules by
compile instructions, and provide batch compiles.

\begin{enumerate}
\def\labelenumi{\arabic{enumi}.}
\item
  \textbf{Software Interfaces and Operation Examples}
\end{enumerate}

A two-way six-lane intersection is built in this research, whose signal
cycle and phase can be adjusted as needed. The setting in Figure 2
represents a signal phase setting of a 90s cycle, with each phase having
45s of green time (green ratio = 50\%). In the interface, we use `E',
`W', `N' and `S' to represent east, west, north, and south respectively,
and use `L', `R' and `C' to represent the three lanes of left, right,
and center respectively (e.g., `WL' represents the left-turn lane from
west to east).

\includegraphics{media/image56.png}

Figure 2 The signal preset interface of the intersection traffic control
simulation system\\
in the connected-vehicle environment

Before simulation, the traffic volume from each direction and the
penetration rate (i.e. the ratio of vehicles that install the speed
guidance equipment) can be preset. The simulation speed can be chosen
from the interface: fast, medium, slow and very slow. In addition, both
the running time and the number of vehicles through each direction can
be monitored. Figure 3 shows that the traffic volume from each direction
is 1080 veh/h, the penetration ratio is 0.3, the simulation speed is
`medium', and the simulation has run for 319.4 seconds.

\includegraphics[width=5.75198in,height=3.06716in]{media/image57.png}

Figure 3 The operation interface of the intersection traffic control
simulation system\\
in the connected-vehicle environment

EXPERIMENTAL RESULTS

We conducted a simulation-based case study using the simulation system
introduced above. The results obtained from the simulation experiments
are presented in this section. Furthermore, we have a discussion on the
differences between the two speed guidance strategies.

\begin{enumerate}
\def\labelenumi{\arabic{enumi}.}
\item
  \textbf{The Effect of Speed Guidance Algorithm on Traffic Efficiency
  under Different Traffic Volumes}
\end{enumerate}

We chose three common measures as indexes of traffic efficiency, i.e.
average delay of vehicles, average number of stops, and average stop
time. To test the effectiveness of the algorithm, the signal cycle is
preset to be 90 s, with a green ratio of 50\% (i.e. each direction has a
green time for 45 s), and the traffic volume from each direction is
preset to be the same, which ranges from 300 to 2700 veh/h (for 300
veh/h intervals) so that the traffic saturation states of low, medium
and high can all be covered.

According to the data we measured at an intersection, when the light
turns to green, the queue clearance speed is about 2.5 s/veh (i.e. 0.4
veh/s). According to this, the saturation traffic volume of three lanes
is calculated as: 3600 s/h ÷ 2 ÷ 2.5 s/veh × 3 = 2160 veh/h.

It should be noted that when the traffic volume is higher than the
saturation volume, vehicles would gradually accumulate in line. As a
results, the collapse of the system will be inevitable. The run time of
the simulation experiment is set to be 3600 s. The aim is to ensure
enough time for the system to operate stably and to avoid the collapse
of the system. Therefore, in the following experiment results, when the
traffic flow is lower than 2160 veh/h, the test results will converge to
a corresponding numerical test result, whereas when the traffic flow is
higher than 2160 veh/h, the test results will be divergent. We only
recorded the run results within 3600 s.

\begin{enumerate}
\def\labelenumi{\arabic{enumi})}
\item
  Analysis of average delay
\end{enumerate}

As stated above, the purpose of speed guidance is to reduce the time to
pass the stop-line. Therefore, the average delay of vehicles is chosen
as the main index to measure the effectiveness of the algorithms. The
delay of a vehicle is defined as the actual time it used to pass the
stop-line minus the virtual time it needs, which is calculated as the
distance divided by the initial velocity.

As shown in Table 2, in the case of no speed guidance, the average delay
increases slowly with the increase in traffic volume when unsaturated;
however, when over saturated (e.g. traffic volume is 2400 veh/h), the
average delay increases sharply as most vehicles have to wait for at
least one signal cycle before they can pass the stop-line (refer to
Figure 4). In the case of single-vehicle speed guidance, when the
traffic volume is lower than 2100 veh/h, the average delay is about 80\%
of that without speed guidance; when the traffic volume is larger than
2400 veh/h, the average delay is about 60\% or even lower of that
without speed guidance; when the traffic volume reaches 2700 veh/h, most
vehicles also have to wait for at least one signal cycle. In the case of
multi-vehicle cooperative speed guidance, the average delay increases
steadily, with no oversaturation. When the traffic volume is lower than
2100 veh/h, the average delay is about 70\% of that without speed
guidance; when the traffic volume is large than 2400 veh/h, the average
delay is only about 20\% or even lower of that without speed guidance.
Therefore, it can be concluded that speed guidance in the
connected-vehicle environment can significantly reduce the average delay
of vehicles. Besides, compared with the strategy of single-vehicle speed
guidance, multi-vehicle cooperative speed guidance is more effective.

Table 2 The average delay of three speed guidance strategies under
different traffic volumes

\begin{longtable}[c]{@{}llllllllll@{}}
\toprule
Traffic volume (veh/h) & 300 & 600 & 900 & 1200 & 1500 & 1800 & 2100 &
2400 & 2700\tabularnewline
\midrule
\endhead
Average delay of vehicles (s) & Without speed guidance & 18.13 & 20.08 &
21.73 & 24.12 & 25.81 & 28.56 & 34.36 & 106.01\tabularnewline
& single-vehicle speed guidance & 14.56 & 16.39 & 18.07 & 19.85 & 21.21
& 23.60 & 28.13 & 40.17\tabularnewline
& multi-vehicle cooperative speed guidance & 12.60 & 14.55 & 16.05 &
17.38 & 18.78 & 20.23 & 21.87 & 23.48\tabularnewline
\bottomrule
\end{longtable}

Figure 4 Comparison of the average delay of three speed guidance
strategies\\
under different traffic volumes

\begin{enumerate}
\def\labelenumi{\arabic{enumi})}
\item
  Analysis of average number of stops
\end{enumerate}

The number of stops can directly affect the average delay of vehicles.
It can also reflect the traffic efficiency. Basing on the analysis
results, we found that the variation pattern of average number of stops
under different traffic volumes (as shown in Table 3 and Figure 5) is
similar to the variation pattern of average delay. In the case of no
speed guidance, the average number of stops rises with the increase in
traffic volume, and remains smaller than 1 when the traffic volume is
unsaturated; however, when traffic volume reaches 2400 veh/h, the
average number of stops rises sharply and becomes larger than 1,
therefore, there is saturation. In the case of single-vehicle speed
guidance, when the traffic volume is lower than 2100 veh/h, the average
number of stops is about 70\% of that without speed guidance; when the
traffic volume is larger than 2400 veh/h, the average number of stops is
only about 40\% or even lower of that without speed guidance. In the
case of multi-vehicle cooperative speed guidance, the average number of
stops is always around 0.2, and it is only about 30\% or even lower of
that without speed guidance.

Table 3 The average number of stops of three speed guidance strategies\\
under different traffic volumes

\begin{longtable}[c]{@{}llllllllll@{}}
\toprule
Traffic volume (veh/h) & 300 & 600 & 900 & 1200 & 1500 & 1800 & 2100 &
2400 & 2700\tabularnewline
\midrule
\endhead
Average number of stops & Without speed guidance & 0.54 & 0.55 & 0.56 &
0.58 & 0.57 & 0.58 & 0.63 & 1.49\tabularnewline
& single-vehicle speed guidance & 0.35 & 0.36 & 0.37 & 0.38 & 0.39 &
0.39 & 0.44 & 0.55\tabularnewline
& multi-vehicle cooperative speed guidance & 0.18 & 0.19 & 0.21 & 0.21 &
0.19 & 0.19 & 0.19 & 0.20\tabularnewline
\bottomrule
\end{longtable}

Figure 5 Comparison of the average number of delays of three speed
guidance strategies\\
under different traffic volumes

\begin{enumerate}
\def\labelenumi{\arabic{enumi})}
\item
  Analysis of average stop time
\end{enumerate}

The average stop time of vehicles is also a common measure of traffic
efficiency at intersections. As shown in Table 4 and Figure 6, through
speed guidance, the average stop time of vehicles decreases. When the
traffic volume is lower than the saturation volume, the optimization
effects of single-vehicle speed guidance and multi-vehicle cooperative
speed guidance are about 20\% and 50\% respectively. When the traffic
volume is higher than the saturation volume (e.g. 2400 veh/h), the
optimization effects of single-vehicle speed guidance and multi-vehicle
cooperative speed guidance are about 50\% and 90\% respectively.

Table 4 The average stop time of three speed guidance strategies\\
under different traffic volumes

\begin{longtable}[c]{@{}llllllllll@{}}
\toprule
Traffic volume (veh/h) & 300 & 600 & 900 & 1200 & 1500 & 1800 & 2100 &
2400 & 2700\tabularnewline
\midrule
\endhead
Average stop time (s) & Without speed guidance & 10.45 & 11.17 & 11.69 &
12.09 & 12.35 & 13.10 & 15.66 & 54.61\tabularnewline
& single-vehicle speed guidance & 8.21 & 8.63 & 9.12 & 9.75 & 10.17 &
11.02 & 13.14 & 20.49\tabularnewline
& multi-vehicle cooperative speed guidance & 5.80 & 5.88 & 5.99 & 6.07 &
6.05 & 6.07 & 6.20 & 6.74\tabularnewline
\bottomrule
\end{longtable}

Figure 6 Comparison of the average stop time of three speed guidance
strategies\\
under different traffic volumes

\begin{enumerate}
\def\labelenumi{\arabic{enumi}.}
\item
  \textbf{The Effect of Penetration Rate on Traffic Efficiency under
  Different Traffic Volumes}
\end{enumerate}

To understand the value of using information from connected vehicles,
simulation tests were conducted by varying the assumed penetration rate
between 0\% and 100\% with 10\% intervals. For these tests, the signal
cycle was set to 90 seconds, with a green ratio of 50\%, and the traffic
volume from each direction being 2160 veh/h. Simulation tests were run
to check the average delay of the two speed guidance strategies under
different penetration rates, with the results shown in Table 5.

The average delay of vehicles resulting from different penetration rates
are shown in Figure 7. For both the single-vehicle speed guidance and
the multi-vehicle cooperative speed guidance algorithms, as the
penetration rate increases, the average delay of vehicles significantly
decreases. However, the marginal benefit obtained from more vehicles
using this technology becomes relatively small after a penetration rate
of 40\% (for both speed guidance strategies). This implies that when the
technology is in the early stage of development (i.e., when the
penetration rate is low), even a few more equipped vehicles can have a
significant effect on reducing the delays of all the vehicles at an
intersection. At these low penetration rates, the information obtained
from each vehicle is very valuable. However, at high penetration rates,
the additional information becomes less and less valuable. Guler,
Menendez and Meier \textsuperscript{{[}21{]}} found similar effects of
penetration rate.

Table 5 The average delay of two speed guidance strategies under
different penetration rates

\begin{longtable}[c]{@{}llllllllllll@{}}
\toprule
Penetration rate & 0\% & 10\% & 20\% & 30\% & 40\% & 50\% & 60\% & 70\%
& 80\% & 90\% & 100\%\tabularnewline
\midrule
\endhead
Average delay of vehicles (s) & single-vehicle speed guidance & 42.50 &
39.87 & 36.01 & 34.69 & 32.16 & 31.46 & 31.31 & 31.32 & 30.28 &
29.69\tabularnewline
& multi-vehicle cooperative speed guidance & 42.64 & 39.83 & 34.94 &
31.40 & 26.39 & 25.87 & 24.72 & 24.39 & 23.83 & 23.08\tabularnewline
\bottomrule
\end{longtable}

Figure 7 The effect of penetration rates on average delay (a volume of
1080 veh/h per direction)

\begin{enumerate}
\def\labelenumi{\arabic{enumi}.}
\item
  \textbf{Comparison of Two Speed Guidance Strategies}
\end{enumerate}

Through the above comparison, we can conclude that speed guidance under
the connected-vehicle environment can significantly improve the
intersection efficiency, and that compared with single-vehicle speed
guidance, multi-vehicle cooperative speed guidance is more effective.

Further analysis implies that the two speed guidance strategies both
reduce the average delay of most vehicles, as Figure 8 shows. As the
traffic volume grows, the multi-vehicle cooperative speed guidance can
significantly move the overall distribution of vehicle delay to the left
(i.e. significantly reduce the delay of most vehicles), thus
significantly lowering the average delay of vehicles at the
intersection.

\includegraphics[width=3.71181in,height=3.10347in]{media/image58.png}

Figure 8 The boxplot of the average delay of three speed guidance
strategies

The reason behind such difference mainly lies in that when
single-vehicle speed guidance is applied, the states of other vehicles
around are unknown, therefore the objective of speed guidance might not
be realized. For example, if the single-vehicle speed guidance algorithm
guides a certain vehicle to run at the speed of 60 km/h in order to
arrive at the intersection before the green light turns to red. However,
the car before it runs at a speed lower than 60 km/h, thus the
speed-guided vehicle has to lower down in order to avoid collision. This
might result in the speed-guided vehicle failing to arrive at the
intersection before the green light turns to red, and having to stop and
wait. As for the multi-vehicle cooperative speed guidance algorithm, it
optimizes all vehicles' time to pass the stop-line, thereby calculating
each vehicle's guided speed, improving both the green time efficiency
and the traffic efficiency.

CONCLUSION

In this study, we have built an intersection traffic-control simulation
platform in the connected vehicle environment, and proposed two speed
guidance algorithms by optimizing the travel time of individual vehicle
and multi vehicles respectively. The goal of optimization is to minimize
the travel time, and common indicators such as average delay of
vehicles, average number of stops, and average stop time are chosen as
indexes of traffic efficiency in simulation experiments.

The simulation results show that compared with the case of no speed
guidance, when the traffic flow is unsaturated (lower than 2160 veh/h),
the single-vehicle speed guidance algorithm can decrease the average
delay by 20\%, the number of stops by 30\%, and the average stop time by
20\%; whereas the multi-vehicle speed guidance algorithm can decrease
the average delay by 30\%, the number of stops by 70\%, and the average
stop time by 50\%. When the traffic flow is saturated (higher than 2160
veh/h), the speed guidance algorithms perform much more outstandingly.
Compared with the case of no speed guidance, the single-vehicle speed
guidance algorithm can decrease the average delay by 40\%, the number of
stops by 60\%, and the average stop time by 50\%; whereas the
multi-vehicle speed guidance algorithm can decrease the average delay by
80\%, the number of stops by 70\%, and the average stop time by 90\%.
With the increase of penetration rate, the effect of speed guidance
gradually enhances, with the most obvious gains obtained when the
penetration rate increases from 10\% to 40\%. However, when the
penetration rate is higher than 60\%, further increase in the
penetration rate has little benefits on the effect of speed guidance
algorithms.

The experimental results indicate that speed guidance in the connected
vehicle environment can significantly improve the traffic efficiency of
intersections, and compared with the single-vehicle speed guidance
strategy, the multi-vehicle speed guidance strategy is more effective.

In the future, we will introduce vehicle-to-vehicle communication into
the simulation system, and realize cooperative driving. In addition, we
will optimize the speed guidance algorithms accordingly for a better
result.

ACKNOWLEDGEMENT

This work was supported in part by the National Natural Science
Foundation of China under Grant 61673233 and 71671100.

REFERENCES

\begin{enumerate}
\def\labelenumi{\arabic{enumi}.}
\item
  L. Li and F.Y. Wang, Cooperative Driving at Blind Crossings using
  Intervehicle Communication, IEEE Transaction on Vehicular Technology,
  Vol. 55, No.6, November 2006.
\item
  L. Li and D.Y. Yao, A Survey of Traffic Control With Vehicular
  Communications, IEEE Transaction on Intelligent Transportation
  Systems, Vol. 15, No.1, November 2014.
\item
  U.S. Department of Transportation.
  https://www.its.dot.gov/cv\_basics/index.htm.
\item
  N. Mohammad and P. Hossein. The effect of VII market penetration on
  safety and efficiency of transportation networks. Proceedings of
  International Conference on Communications Workshops. Dresden: IEEE,
  2009.
\item
  J. Yao, H.Y. Fan, Y. Han, L. Cui. Adaptive Control at Intersection in
  Urban Area Based on Probe Data. Journal of University of Shanghai for
  Science and Technology, 2014, 36(3): 239-244.
\item
  A. H. Booz. Vehicle infrastructure integration (VII) proof-of-concept
  (POC) test---an overview. 2008 ITS VA Annual Conference. Charlottes
  Ville, 2008.
\item
  M. Nekoui. Development of a VII-enabled prototype intersection
  collision warning system. International Journal of Internet Protocol
  Technology, 2009(4):173-181.
\item
  K. J. Malakorn and B. B. Park. Assessment of mobility, energy, and
  environment impacts of intellidrive-based cooperative adaptive cruise
  control and intelligent traffic signal control. Proceedings of the
  2010 IEEE International Symposium on Sustainable Systems and
  Technology, Hilton crystal city: IEEE Xplore, 2010(1):1-6.
\item
  G. Abu-Lebdeh. Exploring the potential benefits of
  intellidrive-enabled dynamic speed control in signalized networks. TRB
  2010 Annual Meeting CD-ROM.
\item
  Y. Yang, S. Chen and J. Sun. Modeling and evaluation of speed guidance
  strategy in VII system. 3th International IEEE Annual Conference on
  Intelligent Transportation Systems, Portugal: IEEE Xplore,
  2010:1045-1050.
\item
  S. Chen, J. Sun and J. Yao. Development and simulation application of
  a dynamic speed dynamic signal strategy for arterial traffic
  management. 4th International IEEE Annual Conference on Intelligent
  Transportation Systems, Washington, D. C., IEEE Xplore,
  2011:1349-1354.
\item
  Q. He. Robust-intelligent traffic signal control within a
  vehicle-to-infrastructure and vehicle-to-vehicle communication
  environment. Tucson: The University of Arizona, 2010.
\item
  J. Lee and B. Park. Development and evaluation of a cooperative
  vehicle intersection control algorithm under the connected vehicles
  environment, IEEE Transactions on Intelligent Transportation Systems,
  13(1): 8190, 2012.
\item
  R.T. Jackline and A.M. Andreas. Automated and Cooperative Vehicle
  Merging at Highway On-Ramps, IEEE Transactions on Intelligent
  Transportation Systems, 2016.
\item
  J.G. Taylor. Hierarchy-of-Models Approach for Aggregation-Force
  Attrition. Proceedings of the Winter Simulation Conference, 2000:
  924-929.
\item
  X. Chen and D. Yao. An empirically comparative analysis of 802.11 n
  and 802.11 p performances in CVIS, ITS Telecommunications (ITST), 2012
  12th International Conference on. IEEE, 2012: 848-851.
\item
  X. Chen, H. Li, C. Yang, et al. An Empirical Analysis of V2I
  Communication in Vehicular Ad-Hoc Network Based on IEEE 802.11n, 19th
  ITS World Congress. 2012.
\item
  X. Chen, D. Yao, Y. Zhang, et al. Design and Implementation of
  Cooperative Vehicle and Infrastructure System Based on IEEE 802.11n.
  Transportation Research Record: Journal of the Transportation Research
  Board, 2011, 2243(1): 158-166.
\item
  W. Leutzbach and R.Wiedemann. Development and applications of traffic
  simulation models at the Karlsruhe Institut für Verkehrswesen. Traffic
  engineering and control, vol. 27, no. 5, pp. 270-278, 1986.
\item
  Regulation on the Implementation of the Road Traffic Safety Law of the
  People's Republic of China. 2004.
\item
  S. I. Guler, M. Menendez and L. Meier. Using connected vehicle
  technology to improve the efficiency of intersections. Transportation
  Research Part C: Emerging Technologies, 2014, 46: 121-131.
\end{enumerate}
